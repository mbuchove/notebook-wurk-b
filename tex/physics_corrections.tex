\documentclass[]{article}

\usepackage{amsmath}
\usepackage{siunitx}
\usepackage{microtype}

\begin{document}
5. This one is ambiguous, it doesn't actually make sense to say whether or not momentum is conserved without defining a system, which they don't do. Given a system, momentum of that system isn't conserved if there are any external forces acting on it. It seems like $A$ is the answer because of the string, not because of the explosion, as you can see in problem 6. $E$ might also make sense, if you count gravity as an external force. 
\newline
\newline
6. Use conservation of momentum with the initial momentum = 0

\begin{align*}
p_i &= p_f \\
	&= 0 \\ 
	&= mv_1 +2mv_2
\end{align*} 

$$ v_1 = -2v_2$$

for speed negative sign doesn't matter
\newline
\newline
7. split this into two parts. first do the collision

$$ p_i = p_f $$
$$ mv_0 = (m+M)v_{block} $$ 
$$ v_{block} = \frac{m}{m+M} v_0 $$

now energy conservation: 

\begin{align}
W &= \Delta K \\
F_f s &= \frac{1}{2}(m+M)v_{block}^2 \\
\mu N s &= \frac{1}{2}(m+M)\left(\frac{m}{m+M} v_0\right)^2 \\ 
\end{align}

$$ N = (m+M)g $$ 
$$ s = \left( \frac{m}{m+M} \right) ^2 \sqrt{\frac{v_0^2}{2\mu g}} $$
\newline
\newline
8. This one is purely definition, elastic means conservation of kinetic energy. 
\newline
\newline
9. same thing here, so $D$ and there's no special name for momentum conservation
\newline
\newline
10. correct
\newline
\newline
11. $A$ because momentum is exchanged between ball and earth. kinetic energy of the ball stays the same according to the problem because speed doesn't change, but this isn't really possible though
\newline
\newline
12. use that result we found to simplify 1D elastic collisions to avoid the quadratic thing:

$$ v_{2i}-v{1i} = -(v_{2f} v_{1f}) $$ 

then combine with standard conservation of momentum:

$$ m_1v_{1i} + m_2v{2i} = m_1v_{1f} + m_2v_{2f} $$ 
\newline
\newline
15. completely inelastic $\rightarrow$ they will stick together. conserve momentum in both x and y:

x:

$$ .1m(45)\sin{\ang{47}} = 1.1mv_{fx} $$ 

y: 

$$ .1m(45)\cos{\ang{47}} + m(15) = 1.1mv_{fy} $$ 

note sin and cos switch from normal because angle is w.r.t. vertical. all signs are positive because everything is either up or to the right. 

now just solve for $v_x$ and $v_y$ then 
$$ \theta = \arctan{\frac{v_x}{v_y}} $$ 
\newline
\newline
16. $4$ is the only possibility, because car One transfers upward momentum to car Two (it loses upward momentum when it rebounds down). Car Two then has its initial rightward momentum and upward momentum, so the answer is $D$
\newline
\newline
17. $E$ -- remember the equation 
$$ \vec{F} = \frac{d\vec{p}}{dt} $$
so if the sum of all forces $\sum \vec{F} = 0 $ then $\frac{d\vec{p}}{dt} = 0 $ thus $\vec{p} $ is constant. in a system, the constituents of the system can move independently, so the overall momentum of a system can be described as the motion or momentum of its center of mass
\newline
\newline
19. use conservation of momentum, only $y$ is necessary:
$$2(1)(10)\cos \ang{60} = 2v_y $$ 
$$ v_y = 5 $$ 
\newline
\newline
20. $A$ -- \SI{0}{m \per s}

the initial momentum of the system was $0$, and an explosion is an internal interaction, and no external forces acted on it to change its momentum, so its final momentum is $0$
\newline
\newline
Free response 1. 

Your approach is right, just need a negative sign for the \SI{4}{g} mass because they say it is moving in the $-x$ direction
\newline
\newline
Free response 2. 

again, right idea, but you only found the x-position, this is 2-dimensional, so just do the same thing with the y-coordinates to find $y_cm$

Free response 3. 

more conservation of momentum, a little bit harder mathematically but same concepts. do x and y: 

x:
$$ (0.025)(5.5) = (0.025)v_A \cos{\ang{65}} + (0.05)v_B \cos{\ang{37}} $$ 

y:
$$ 0 = (0.025)v_A \sin{\ang{65}} - (0.05)v_B \sin{\ang{37}} $$ 

easiest way to solve system of equations here is to solve $y$ equation for either $v_A$ or $v_B$ and plug into $x$ equation! 

\end{document}
