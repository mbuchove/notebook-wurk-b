\documentclass{article}


\begin{document}

\hfill Matt Buchovecky

\hfill Stats C283

\hfill Stock Portfolio Project \hfill 
\\ \\ 

For this project, I chose 5 different stocks from each of 5 industries: technology, consumer goods, healthcare, services, and financial. I chose many of the stocks that I was interested in, had data since the beginning of 2008, and chose stocks that did not pay out regular dividends. I used the stockPortfolio package to perform my analysis on the stocks. After digital retrieval of their data, I analysed their historical data, which consisted of their monthly return between 2008 and 2013. Using that historical data, I found the optimal portfolio by various methods: the Markowitz model, single index model, constant correlation model, and multi group model. I also compared the optimal portfolios for the aforementioned models both when short selling is allowed and disallowed. 

My report includes two primary plots and one table. 	The first plot is a standard return plotted against risk. The portfolio possibilities curve and capital allocation line are shown, as calculated using the Markowitz method. Also included are the points for individual stocks, shown as blue circles. The yellow diamonds represent the optimal portfolios as calculated by the various aforementioned models, still restricted to historical data. The Single Index Model yielded the lowest risk, with even less risk when short selling was allowed.  

The next plot shows the cumulative return on a month by month basis between 2013 and the present date. All stocks portfolios performed better than the Standard \& Poor market average in the long term, even though they performed worse than it in the first 5 - 10 months. The least impressive portfolio, with funds allocated equally between all stocks, outperformed the S\&P by a few percent. The next best was the mix of half of the previous portfolio, and half single index model, which was outperformed by the pure single index portfolio. The best two portfolios were those of the multi group model and constant correlation model. The multi group portfolio reached the highest return, but dropped quickly in the last 5 - 6 months. 

Table \ref{tabel} summarizes the measures of the performance of each portfolio using many measurements. The average return was consistent with the ranking of the cumulative return mentioned above, although it is possible for these two to differ. The highest values for the Sharpe ratio were the single index model and constant correlation model, while the multi group model had a relatively low value despite performing well in return. The Treynor measure and Jensen measure both followed the same ranking as the pure returns, with the constant correlation model yielding the best value. The differential return value was almost the same, with the single index model slightly exceeding the multi group model. 

Overall, the constant correlation model gave the strongest portfolio, and my stocks gave very impressive year, giving up to 50\% growth in under 4 years! 



\begin{table}[b]
 \centering
  \begin{tabular}{ | c | c | c | c | c | c | }
   \hline 
   \multicolumn{6}{|c|}{Summary} \\ \hline 
  % Model & Sharpe Ratio  & Jensen Measure & Treynor Measure & Differential return \\ \hline 
   
   Model & Equal All. & Single Index & Half and half & Const. corr. & Multi group \\ \hline
   
   Avg return & 0.008872291 & 0.012830804 & 0.010851547 & 0.016624435 & 0.013639519 \\ \hline 
   
   Sharpe ratio & 6.160243 & 9.775235 & 8.531433 & 10.940561 & 2.413005 \\ \hline

   Treynor Measure & 0.01132425 & 0.01765951 & 0.01443333 & 0.02412854 & 0.01836149  \\ \hline 
   
   Jensen measure & 0.003350412 & 0.007473050 & 0.005411731 & 0.011412317 & 0.008161872 \\ \hline 
   
   Diff return & 0.007475295 & 0.011454819 & 0.009492819 & 0.015180777 &  0.011012265  \\ \hline 

  \end{tabular}
 \caption{Summary of the performance of the }  
 \label{tabel}
\end{table}


\end{document}
